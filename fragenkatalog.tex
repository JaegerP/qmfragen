%Fragenkatalog
\documentclass[12pt,a4paper,notitlepage]{report}

%\usepackage[T1]{fontenc}
\usepackage[utf8]{inputenc}
\usepackage[ngerman]{babel}
\usepackage{amsmath,xfrac,mathtools}
\usepackage{amsfonts,mathrsfs}
\usepackage{amssymb}
\usepackage{euscript,xspace}
\usepackage{graphicx}
\usepackage{geometry}
\usepackage{amsthm}
\usepackage{float,sidecap}
\usepackage{fancyhdr}
\usepackage{cancel}
\usepackage{paralist}
\usepackage{wrapfig} %hinzugefügt
\usepackage{relsize}
\usepackage{environ}
\setcounter{tocdepth}{1}
\newcommand{\theauthor}{Philipp Jaeger}
\author{\theauthor}
\date{}

\newcommand{\thetitle}{Fragenkatalog zu Quantenmechanik 1\\\smaller{Prof. Eggert, Sommersemester 2013}}
\title{\textbf{\thetitle}\vspace{25mm}}

\geometry{a4paper,left=25mm,right=25mm, top=15mm, bottom=55mm}
%\setlength{\mathindent}{15mm}

\newif\ifanswers
\answerstrue % comment out to hide answers

\fancyhead{}
\fancyhead[R]{\textbf{\thetitle}\\\theauthor}
\fancyfoot[C]{\thepage}
\renewcommand{\headrulewidth}{0.4pt}
\renewcommand{\footrulewidth}{0pt}
\addtolength{\headheight}{70.5pt}
\pagestyle{fancy}

\renewcommand{\familydefault}{\sfdefault}
%hätte gerne nen sf-font

\newcommand{\betrag}[1]{\ensuremath {\mid #1 \mid}}
\newcommand{\dif}{\mathsf{d}}
\newcommand{\logn}{\mathsf{ln}}
\newcommand{\partd}[1]{\ensuremath {\frac{\partial}{\partial #1}}}
\newcommand{\partdd}[2]{\ensuremath {\left(\frac{\partial #1}{\partial #2}}\right)}
\newcommand{\partddd}[3]{\ensuremath {\left(\frac{\partial #1}{\partial #2}}\right)_{#3}}
\newcommand{\pseq}{\mathrel{\phantom{=}}}
\newcommand{\binkoef}[2]{\left(\begin{array}{c}{#1}\\{#2}\end{array}\right)}
\newcommand{\const}{\mathsf{const.}}
\newcommand{\ket}[1]{\ensuremath {\mid #1 \rangle}}
\newcommand{\bra}[1]{\ensuremath {\langle #1 \mid}}

\let\oldoverset\overset
\renewcommand{\overset}[2]{\oldoverset{\mathclap{#1}}{#2}\quad}

\newtheorem{mydef}{Definition}
%\newtheorem{myfrag}{}%Frage} doppelt nur die Nummerierung
\NewEnviron{myfrag}[1]{\begin{it}#1\end{it}\ifanswers\par\expandafter\BODY\fi}
%\NewEnviron{foobar}{\iffoo\expandafter\BODY\fi}
\newtheorem{mybem}{Bemerkung}
\newtheorem{mybei}{Beispiel}
\newtheorem{mybeh}{Behauptung}
\newtheorem{mybew}{Beweis}
\newtheorem{mycod}{Code}%sollte man als environment mit monospace bauen

\renewcommand{\thesubsection}{Frage}
\numberwithin{equation}{section}

\begin{document}
\maketitle% sollte man als header verbauen

%\clearpage
%\tableofcontents%macht mir zu viel Gedöns, hätte aber gerne thematische Sortierung in sections
%fragen 1-9
\vspace{50mm}
\begin{abstract}
\par Die vorliegende Fragensammlung besteht aus den mit den Übungsblättern verteilten Verständnisfragen zur Vorlesung \glqq Quantenmechanik 1\grqq\ aus dem Sommersemester 2013 von Prof. Eggert, TU Kaiserslautern.
\par Sie wurde als Hilfsmittel zur Klausur- und Prüfungsvorbereitung von Studenten für Studenten geschrieben und keineswegs fehlerfrei, eventuelle inhaltliche Fehler werden aber gerne korrigiert. Rechschreibfehler dürfen vom Finder behalten werden.
\end{abstract}

%Fragen 10-19
\subsection{1}
\begin{myfrag}{Eine Frage.}
Und die Antwort:
\end{myfrag}

%Fragen 20-29
\input{2.tex}

%Fragen 30-39
\input{3.tex}

%Fragen 40-49
\input{4.tex}

%Fragen 50-59
\input{5.tex}

%Fragen 60-69
\input{6.tex}

%Fragen 70-79
\input{7.tex}

%Fragne 80-89
\input{8.tex}

%Fragen 90-99
\input{9.tex}

%Fragen 100-109
\input{10.tex}

\end{document}